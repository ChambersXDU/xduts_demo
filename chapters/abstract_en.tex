This study examines the integration of cattle and horse power into modern civil engineering, drawing on their historical use in plastering techniques. It highlights the biomechanical advantages of these animals, such as efficiency and low energy consumption, and their potential to inspire modern crane design. The paper also addresses engineering ethics, emphasizing animal welfare and the sustainability of using animal power in construction. Experimental findings show that in complex terrains, animal power can significantly reduce construction time and costs while minimizing environmental impact. The research further explores the application of animal dynamics in remote areas with limited infrastructure, where it offers economic and environmental benefits. The study also considers how animal behavior and biomechanics can optimize construction processes and enhance structural stability.